%-------------------------------------------------------------------------------
%	SECTION TITLE
%-------------------------------------------------------------------------------
\cvsection{Projects and Experience}


%-------------------------------------------------------------------------------
%	CONTENT
%-------------------------------------------------------------------------------
\begin{cventries}

%---------------------------------------------------------
  \cventry
    {Project under Prof. Swaprava Nath} % Job title
    {Swagrader: Honest Effort Extracting, Modular Peer-Grading Tool} % Organization
    {IIT Kanpur, India} % Location
    {May 2019-present} % Date(s)
    {
      \begin{cvitems} % Description(s) of tasks/responsibilities
        \item {Built a customizable and game theoritically powered tool \textbf{from scratch} which handles peer grading and normal grading workflows for massive open online courses.}
        \item {The core of the tool is powered by \textbf{TRUPEQA} (TRUthful Peer Evaluation with Quality Assurance), which is a mechanism devised to extract honest effort grading from the strategic peers. By its properties viz. \textit{EBPI} and \textit{EPRM} the algorithm has a stronger version of similar definition of \textbf{dominant strategies}.}
        \item {Contributed a more robust distribution alogrithm which essentially is the \textbf{graph connectivity} problem.}
        \item {Worked with extensive data management and design patterns to handle big data for the students and handle accordingly.}
        \item {Used \textbf{cryptographic signing} for anonymous peergrading.}        
        \item {A demonstration of the tool was presented in \textit{CoDS-COMAD}. The paper on the tool is \textbf{one of the 6} accepted papers in the conference.}        
        \item {The tool will come live next summers \textbf{campus wide} for beta testing.}
      \end{cvitems}
    }

%---------------------------------------------------------
\cventry
    {1st Runner-Up, Dev-Hackathon, Inter-IIT TechMeet} % Affiliation/role
    {SRLMS: Smart Road Lease and Management System} % Organization/group
    {IIT Roorkee, India} % Location
    {December 2019} % Date(s)
    {
      \begin{cvitems} % Description(s) of experience/contributions/knowledge
        \item {Developed an app based on Restful API backend called SRLMS (smart road lease and maintenance system), which can be used effectively by the government, contractors, and citizens to improve the condition of roads.}
        \item {The principle underlying is \textbf{crowdsourcing}. The Modular reward system was based on game theoritic arguments, to ensure quality of data reddit based voting system was implemented. Location based APIs were used to filter the complaints for one municipal office, moreover, real time location features and acceleratometer features were used for more robust reward system called as \textbf{Run and Earn}.}        
        \item {The frontend was developed with React Native (app) and Django templates (web app). Server-side based on Django Rest Framework, exposing tens of APIs for any modular frontend.}        
        \item {Fully \textbf{dockerized} backend running on a docker-compose setup. Written with scalability and speed in mind.}
        \item {\textbf{Judged 2nd} among all IITs participating in the competition.}
      \end{cvitems}
    }
    
\cventry
    {Summer of Code, IIT Kanpur} % Affiliation/role
    {Saarthi: Smart Road Maintenance system} % Organization/group
    {IIT Kanpur, India} % Location
    {May-Jun 2019} % Date(s)
    {
      \begin{cvitems} % Description(s) of experience/contributions/knowledge
        \item {Developed an app based on Restful API backend called Saarthi, which can be used by the government and citizens to maintain roads.}
        \item {Crowdsourcing was the idea for the project. \textbf{Location based tagging} and implementing crpytographic system in backend was used.}
        \item {The frontend was developed with \textbf{Flutter} (app) and Django templates (web app). Server-side based on \textbf{Django Rest Framework}, exposing tens of APIs for any modular frontend.}
      \end{cvitems}
    }

\cventry
    {Association of Computer Activities, Department of Computer Science and Engineering, IIT Kanpur} % Affiliation/role
    {LWCPM: Light Wieght and Custom Package Manager} % Organization/group
    {IIT Kanpur, India} % Location
    {Jan 2019 - Mar 2019} % Date(s)
    {
      \begin{cvitems} % Description(s) of experience/contributions/knowledge
        \item {Developed a Python based package manager that installs custom packages and dependencies from linux source repositories.}
        \item {To validate the package, \textbf{RSA signatures} were used in packages.}
        \item {Implemented \textbf{graph algorithms} to tackle package dependencies and circular dependencies.}
        \item {Implemented \textbf{threads} and \textbf{file locks} to speed up the installation and dependency list making process.}
      \end{cvitems}
    }

%---------------------------------------------------------
\end{cventries}
